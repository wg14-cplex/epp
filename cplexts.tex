% Use of LuaTeX seems to be recommended.
% This stuff is all about "standard" LaTex capabilities used in the document.
\documentclass[
	techspec,	% for a Technical Specification
	final,		% needed for PDF bookmarks
	notcopyright,	% no copyright statements (yet)
	letterpaper	% for now
	]{isov2}
\let\ifpdf\relaxed	% let hyperref implement \ifpdf
\usepackage[
	pdftitle={Extensions for parallel programming},
	pdfauthor={Clark Nelson},
	pdfsubject={WORKING DRAFT Technical Specification},
	pdfkeywords={},
	pdfstartview=FitH,
	colorlinks=true, % not boxed
	linkcolor=blue,	% usual default for browsers
	urlcolor=blue
	]{hyperref}	% for PDF hyperlinks and bookmarks
\usepackage{color}	% for editorial marking
\usepackage{ifthen} 	% for conditional code/text
\usepackage{listings}	% for code listings
\usepackage{makeidx}	% for indexing
\usepackage{ulem}	% for underlining and strikeout
\usepackage{underscore} % to simplify use of underscores
\usepackage{verbatim}	% for code font
\usepackage{xspace}	% to simplify a macro that expands to text
\makeindex
\setcounter{secnumdepth}{6}
\setcounter{tocdepth}{6} % increase to check outline structure

% These macros are from the WG21 standard source.
\input{macros}
\renewcommand{\indextext}[1]{\index{#1}}
\renewcommand{\indexgram}[1]{\index{#1}}

% This is for macros invented for/by CPLEX.
\newboolean{hidecpp}
\setboolean{hidecpp}{false}
\ifthenelse{\boolean{hidecpp}}
{\let\cpp=\comment
\let\endcpp=\endcomment
\newcommand{\nocpp}[1]{#1}
\newcommand{\yescpp}[1]{}
}
{\newenvironment{cpp}
{\sffamily\slshape[C++:}
{]}
\newcommand{\nocpp}[1]{}
\newcommand{\yescpp}[1]{#1}
}


% These parameters, and the documentclass options,
% are about document identification and overall formatting.
\newcommand{\cplexts}{$CPLEXTS$}
\renewcommand{\extrahead}{2014-02-05}
\standard{\cplexts}
\yearofedition{2016}
\languageofedition{(E)}

\begin{document}

\begin{cover}
This is the cover page.
When ISO publishes a document, they apply their own title page,
so CPLEX/WG14 can put anything here.
\clearpage
\end{cover}

% The table of contents appears here.

\begin{foreword}
ISO (the International Organization for Standardization) is a worldwide federation of national standards
bodies (ISO member bodies). The work of preparing International Standards is normally carried out
through ISO technical committees. Each member body interested in a subject for which a technical
committee has been established has the right to be represented on that committee. International organizations,
governmental and non-governmental, in liaison with ISO, also take part in the work. ISO
collaborates closely with the International Electrotechnical Commission (IEC) on all matters of electrotechnical
standardization.

International Standards are drafted in accordance with the rules given in the ISO/IEC Directives, Part 2.

The main task of technical committees is to prepare International Standards. Draft International Standards
adopted by the technical committees are circulated to the member bodies for voting. Publication
as an International Standard requires approval by at least 75\% of the member bodies casting a vote.

In other circumstances, particularly when there is an urgent market requirement for such documents, a
technical committee may decide to publish other types of normative document:

\begin{itemize}
\item
an ISO Publicly Available Specification (ISO/PAS) represents an agreement between technical experts
in an ISO working group and is accepted for publication if it is approved by more than 50\%
of the members of the parent committee casting a vote;
\item
an ISO Technical Specification (ISO/TS) represents an agreement between the members of a technical
committee and is accepted for publication if it is approved by 2/3 of the members of the
committee casting a vote.
\end{itemize}

An ISO/PAS or ISO/TS is reviewed every three years with a view to deciding whether it can be transformed
into an International Standard.


ISO/TS
\cplexts{}
was prepared by Technical Committee ISO/IEC JTC1/SC22/WG14.%
\footnote{EN:
This is the only paragraph in the Foreword that has anything in it
that's not just boilerplate.
}

\fwdnopatents
\end{foreword}

\begin{introduction}
\intropatents
\end{introduction}

\title{Programming languages}{C}{Extensions for parallel programming}

\scopeclause
\begin{inscope}{technical specification}
\item
Extensions to the C language to simplify parallel programming.%
\footnote{EN:
This is just my first guess.
}
\end{inscope}
\begin{outofscope}{technical specification}
\item
World peace.%
\footnote{EN:
A pretty safe bet.
}
\end{outofscope}

\normrefsclause
\normrefbp{technical specification}
\begin{nreferences}
\isref{ISO/IEC 9899:2011(E)}{Programming languages --- C}
\disref{ISO/IEC 14882:2014(E)}{Programming languages --- C++}
\end{nreferences}

\defclause
\begin{definitions}

For the purposes of this document,
the following terms and definitions apply.

\definition{thread of execution}{
flow of control within a program,
including a top-level statement or expression,
and recursively including every function invocation it executes%
\footnote{FYI:
Adapted from the C++ standard.
}
}

\definition{OS thread}{
service provided by an operating system,
which can be invoked using the
\tcode{thrd_create} function,
for executing multiple threads of execution concurrently
}

\begin{note}
There is typically significant overhead involved
in creating a new OS thread.
\end{note}

\definition{thread}{
thread of execution, or OS thread
}

\begin{note}
This word, when used without qualification, is ambiguous.
\end{note}

\definition{execution agent}{
entity, such as an OS thread,
that may execute a thread of execution
in parallel with other execution agents%
\footnote{FYI:
Adapted from the C++ standard.
}
}

\definition{task}{
thread of execution within a program
that can be correctly executed asynchronously
with respect to
independent threads of execution from}
the program

\definition{concurrent program}{
program that uses multiple concurrent interacting threads of execution,
each with its own progress requirements
}

\begin{example}
A program that has separate server and client threads
is a concurrent program.
\end{example}

\definition{parallel program}{
program whose computation
involves independent tasks,
which may be distributed across multiple computational units
to be executed simultaneously
}

\begin{note}
If sufficient computational resources are available,
a parallel program may execute significantly faster than
an otherwise equivalent serial program.
\end{note}

\end{definitions}


\clause{Document conventions}

\begin{cpp}
Text that is specific to C++
is enclosed in square brackets
and presented in oblique sans-serif type.
\end{cpp}

Definitions of terms and grammar non-terminals defined in the C
\begin{cpp}
or C++
\end{cpp}
standard are not duplicated in this document.
Terms and grammar non-terminals defined in this document
are referenced in the index.

According to the ISO editing directives,
the use of footnotes
``shall be kept to a minimum.''
Almost all of the footnotes in this document
are not intended to survive to final publication.
Most footnotes are classified by an abbreviation:
\begin{description}
\item[EN]
Editor's note.
These mostly call attention to an area that needs more work.
\item[DFEP]
Departure from existing practice.
\end{description}

Annex A contains information
concerning the editing of the LaTeX source of this document.
It will not survive to final publication.

Despite the examples of the C and C++ standards,
the ISO editing directives state quite clearly
that paragraphs should not be numbered.
(``A paragraph is an unnumbered subdivision of a clause or subclause.'')
Consequently, the LaTeX document class used for this document,
which is specifically intended for ISO documents,
does not support paragraph numbering.

\clause{Counted loops}
\sclause{Introduction}
\pnum
A
\defn{counted loop}
is a
\tcode{for}
statement
\begin{cpp}
or range-based
\tcode{for}
statement
\end{cpp}
that is required to satisfy additional constraints.
The purpose of these constraints is to ensure that
the loop's iteration count can be computed
before the loop body is executed.

\pnum
There shall be no
\tcode{return},
\tcode{break},
\tcode{goto}
or
\tcode{switch}
statement that might transfer control into or out of a counted loop.

\sclause{Constraints on a counted \tcode{for} statement}
\ssclause{Introduction}

\pnum
The syntax of a
\tcode{for}
statement includes three
\defn{control clauses}
between parentheses,
separated by semicolons.
The first of these is called the initialization clause;
the second is called the condition clause or controlling expression;
the third is called the
\defn{loop-increment}.

\pnum
When a constraint limits the form of an expression,
parentheses are allowed around the expression
or any required subexpression.

\ssclause{Constraints on the form of the control clauses}

\pnum
\begin{cpp}
The
\nonterminal{condition}
shall be an expression.

\begin{note}
A condition with declaration form is useful
in a context where a value carries more information
than just whether it is zero or nonzero.
This is not believed to be useful in a counted loop.
\end{note}
\end{cpp}

\pnum
The controlling expression shall be a comparison expression
with one of the following forms:%
\footnote{DFEP:
OpenMP does not (yet) allow comparison with
\tcode{!=}.
}

\begin{bnf}
\br
relational-expression \terminal{<} shift-expression
\br
relational-expression \terminal{>} shift-expression
\br
relational-expression \terminal{<=} shift-expression
\br
relational-expression \terminal{>=} shift-expression
\br
equality-expression \terminal{!=} relational-expression
\end{bnf}

\pnum
Exactly one of the operands of the comparison operator
shall be an identifier designating an induction variable,
as described below.
This induction variable is known as the
\defn{control variable}.
The operand that is not the control variable is called the
\defn{limit expression}.
\begin{cpp}
Any implicit conversion applied to that operand is not considered
part of the limit expression.
\end{cpp}

\pnum
The loop-increment shall be an expression with the following form:%
\footnote{DFEP:
OpenMP and ``classic'' Cilk allow only a single induction variable:
the loop control variable.
Allowing multiple induction variables is implemented in Intel's compiler.
}

\begin{bnf}
\nontermdef{loop-increment}
\br
single-increment
\br
loop-increment \terminal{,} single-increment
\end{bnf}

\begin{bnf}
\nontermdef{single-increment}
\br
identifier \terminal{++}
\br
identifier \terminal{--}
\br
\terminal{++} identifier
\br
\terminal{--} identifier
\br
identifier \terminal{+=} initializer-clause
\br
identifier \terminal{-=} initializer-clause
\br
identifier \terminal{=} identifier \terminal{+} multiplicative-expression
\br
identifier \terminal{=} identifier \terminal{-} multiplicative-expression
\br
identifier \terminal{=} additive-expression \terminal{+} identifier
\end{bnf}

\pnum
\begin{cpp}
Each comma in the grammar of loop-increment shall represent
a use of the built-in comma operator.
\end{cpp}
The identifier in each grammatical alternative for single-increment
names an
\defn{induction variable}.
If
\grammarterm{identifier}
occurs twice in a grammatical alternative for
\grammarterm{single-increment},
the same variable shall be named by both occurrences.
If a grammatical alternative for
\grammarterm{single-increment}
contains a subexpression
that is not an identifier for the induction variable,
that is called the
\defn{stride expression}
for that induction variable.

\pnum
An induction variable shall not be designated by more than one
\grammarterm{single-increment}.

\begin{note}
The control variable is identified by considering
the loop's condition and loop-increment together.
If exactly one operand of the condition comparison is a variable,
it is the control variable, and must be incremented.
If both operands of the condition comparison are variables,
only one is allowed to be incremented;
that one is the control variable.
It is an error if neither operand of the condition comparison is a variable.
\end{note}

\begin{note}
There is no additional constraint on the form
of the initialization clause of a counted \tcode{for} loop.%
\footnote{DFEP:
OpenMP and ``classic'' Cilk require that the control variable be initialized.
This relaxation is implemented in Intel's compiler.
}
\end{note}

\ssclause{Other statically checkable constraints}

\pnum
Each induction variable shall have unqualified integer%
\yescpp{,}
\begin{cpp}
enumeration,
copy-constructible class,
\end{cpp}
or pointer type,
and shall have automatic storage duration.

\pnum
Each stride expression shall have integer
\begin{cpp}
or enumeration
\end{cpp}
type.

\newcommand{\MYin}{\hspace{0.1in}}
\newcommand{\MYcolA}{0.7in}
\newcommand{\MYcolB}{1.0in}
\newcommand{\MYcolC}{1.3in}

\begin{table}[ht]
\caption{%
Method of computing the iteration count
}
\label{tab:itcount}
\centering
\begin{tabular}{|l|l|l|l|l|}
\hline
\parbox[c][30pt]{\MYcolA}{
\bfseries
Form of\\condition
}&
\multicolumn{4}{c|}{
\bfseries
Form of single-increment
}
\\ \hline &
\parbox{\MYcolB}{
\bfseries
\grammarterm{id} \tcode{++}\\
\tcode{++} \grammarterm{id}
}&
\parbox{\MYcolB}{
\bfseries
\grammarterm{id} \tcode{--}\\
\tcode{--} \grammarterm{id}
}&
\parbox[c][40pt]{\MYcolC}{
\bfseries
\grammarterm{id} \tcode{+=} \grammarterm{stride}\\
\grammarterm{id} \tcode{=} \grammarterm{id} \tcode{+} \grammarterm{stride}\\
\grammarterm{id} \tcode{=} \grammarterm{stride}\tcode{+} \grammarterm{id}
}&
\parbox{\MYcolC}{
\bfseries
\grammarterm{id} \tcode{-=} \grammarterm{stride}\\
\grammarterm{id} \tcode{=} \grammarterm{id} \tcode{-} \grammarterm{stride}
}
\\ \hline
\parbox[c][30pt]{\MYcolA}{
\bfseries
\grammarterm{id} \tcode{<} \grammarterm{lim}\\
\grammarterm{lim} \tcode{>} \grammarterm{id}
}&
$((lim)-(id))$&
ERROR&
\parbox{\MYcolC}{
$((lim)-(id)-1)/$\\
\MYin$(stride)+1$
}&
\parbox{\MYcolC}{
$((lim)-(id)-1)/$\\
\MYin$(stride)+1$
}
\\ \hline
\parbox[c][30pt]{\MYcolA}{
\bfseries
\grammarterm{id} \tcode{>} \grammarterm{lim}\\
\grammarterm{lim} \tcode{<} \grammarterm{id}
}&
ERROR&
$((id)-(lim))$&
\parbox{\MYcolC}{
$((id)-(lim)-1)/$\\
\MYin$-(stride)+1$
}&
\parbox{\MYcolC}{
$((id)-(lim)-1)/$\\
\MYin$-(stride)+1$
}
\\ \hline
\parbox[c][30pt]{\MYcolA}{
\bfseries
\grammarterm{id} \tcode{<=} \grammarterm{lim}\\
\grammarterm{lim} \tcode{>=} \grammarterm{id}
}&
\parbox{1in}{
$((lim)-(id))$\\
\MYin+1
}&
ERROR&
\parbox{\MYcolC}{
$((lim)-(id))/$\\
\MYin$(stride)+1$
}&
\parbox{\MYcolC}{
$((lim)-(id))/$\\
\MYin$(stride)+1$
}
\\ \hline
\parbox[c][30pt]{\MYcolA}{
\bfseries
\grammarterm{id} \tcode{>=} \grammarterm{lim}\\
\grammarterm{lim} \tcode{<=} \grammarterm{id}
}&
ERROR&
\parbox{\MYcolB}{
$((id)-(lim))$\\
\MYin$+1$
}&
\parbox{\MYcolC}{
$((id)-(lim))/$\\
\MYin$-(stride)+1$
}&
\parbox{\MYcolC}{
$((id)-(lim))/$\\
\MYin$-(stride)+1$
}
\\ \hline
\parbox{\MYcolA}{
\bfseries
\grammarterm{id} \tcode{!=} \grammarterm{lim}\\
\grammarterm{lim} \tcode{!=} \grammarterm{id}
}&
$((lim)-(id)$&
$((id)-(lim))$&
\parbox[c][72pt]{\MYcolC}{
$((stride)<0)$ \tcode{?}\\
\MYin$((id)-(lim)-1)/$\\
\MYin\MYin$-(stride)+1$ \tcode{:}\\
\MYin$((lim)-(id)-1)/$\\
\MYin\MYin$(stride)+1$
}&
\parbox{\MYcolC}{
$((stride)<0)$ \tcode{?}\\
\MYin$((lim)-(id)-1)/$\\
\MYin\MYin$-(stride)+1$ \tcode{:}\\
\MYin$((id)-(lim)-1)/$\\
\MYin\MYin$(stride)+1$
}
\\ \hline
\end{tabular}
{\bfseries
Legend:
}

\begin{tabular}{|l|l|l|}
\hline
\bfseries Name&
\bfseries In the form of an expression&
\bfseries In the iteration count expression
\\ \hline
$id$&
The name of the control variable.&
\parbox[c][30pt]{2.5in}{
An expression with the type and value\\
of the control variable.
}
\\ \hline
$lim$&
The limit expression.&
\parbox[c][30pt]{2.5in}{
An expression with the type and value\\
of the limit expression.
}
\\ \hline
$stride$&
The stride expression.&
\parbox[c][40pt]{2.5in}{
An expression with the type and value\\
of the stride expression
for the control variable.
}
\\ \hline
\end{tabular}
\end{table}

\pnum
The
\defn{iteration count}
is computed according to
\tref{tab:itcount}.%
\footnote{EN:
We need to say something about error cases.
}
The iteration count is computed after the loop initialization is performed,
and before the control variable is modified by the loop.
\begin{cpp}
The iteration count expression shall be well-formed.
\end{cpp}

\pnum
The type of the difference between the limit expression and the control variable
is the
\defn{subtraction type}\nocpp{.}%
\yescpp{,}
\begin{cpp}
which shall be integral.
When the condition operation is !=,
(limit)-(var) and (var)-(limit) shall have the same type.
\end{cpp}
Each stride expression shall be convertible to the subtraction type.
\begin{cpp}
The loop odr-uses whatever operator-functions are selected
to compute these differences.
\end{cpp}

\begin{cpp}

\begin{table}[ht]
\caption{%
Method of advancing an induction variable
}
\label{tab:inc}
\centering
\begin{tabular}{|l|l|}
\hline
\bfseries Single-increment operator&
\bfseries Expression
\\ \hline
\tcode{++ += +}&
$V$ \tcode{+=} $X$
\\ \hline
\tcode{-- -= -}&
$V$ \tcode{-=} $X$
\\ \hline
\end{tabular}
\end{table}

\pnum
For each induction variable
$V$,
one of the expressions from
\tref{tab:inc}
shall be well-formed,
depending on the operator used in its single-increment.
In the table,
$X$
stands for some expression with the same type as the subtraction type.
The loop odr-uses whatever
\tcode{operator+=}
and
\tcode{operator-=}
functions are selected by these expressions.
\end{cpp}

\ssclause{Dynamic constraints}

\pnum
If an induction variable is modified within the loop
other than as the side effect of its single-increment operation,
the behavior of the program is undefined.%

\begin{cpp}
If evaluation of the iteration count,
or a call to a required
\tcode{operator+=}
or
\tcode{operator-=}
function,
terminates with an exception,
the behavior of the program is undefined.
\end{cpp}

\pnum
If $X$ and $Y$ are values of the control variable
that occur in consecutive evaluations of the loop condition
in the serialization,
then the behavior is undefined if
$((limit) - X) - ((limit) - Y)$,
evaluated in infinite integer precision,
does not equal the stride.

\begin{note}
In other words, the control variable must obey the rules of normal arithmetic.
Unsigned wraparound is not allowed.
\end{note}

\pnum
If the condition expression is true on entry to the loop,
then the behavior is undefined
if the computed iteration count is not greater than zero.
If the computed iteration count is not representable
as a value of type
\tcode{unsigned long long},
the behavior is undefined.

\ssclause{Evaluation relaxations}

\pnum
The stride expressions shall not be evaluated if the iteration count is zero;
otherwise,
the stride and limit expressions are evaluated exactly once.%
\footnote{DFEP:
Neither OpenMP nor Cilk specifies
how many times these expressions must be evaluated.
}
%If execution of a loop iteration alters the value
%of the increment or limit expression,
%the behavior is undefined.

\pnum
Within each iteration of the loop body,
the name of each induction variable refers to a local object,
as if the name were declared as an object within the body of the loop,
with automatic storage duration and with the type of the original object.
\begin{cpp}
If the loop body throws an exception
that is not caught within the same iteration of the loop,
the behavior is undefined, unless otherwise specified.
\end{cpp}

\begin{cpp}
\sclause{Constraints on a counted range-based \tcode{for} statement}
\pnum
In a counted range-based
\tcode{for}
statement ([stmt.ranged] 6.5.4),
the type of the
\tcode{__begin}
variable,
as determined from the
\nonterminal{begin-expr},
shall satisfy the requirements of a random access iterator.
\begin{note}
Intel has not yet implemented support for
a parallel range-based
\tcode{for}
statement.
\end{note}
\end{cpp}


\infannex{LaTeX samples}
\sclause{Block-level normative elements}
\ssclause{\ttfamily\textbackslash begin\{enumerate\}}
\begin{enumerate}
\item
This is an enumerated list.
\begin{enumerate}
\item
This is a nested enumerated list.
\end{enumerate}
\end{enumerate}

\ssclause{\ttfamily\textbackslash begin\{description\}}
\begin{description}
\item[term]
This is a description list.
\item[longer term]
More description list.
\end{description}

\ssclause{\ttfamily\textbackslash begin\{bnf\}}
Grammar rules appear in the index automatically.
The ``opt'' suffix should be applied only to the name of a non-terminal.

\begin{bnf}
\nontermdef{sample-rule-name}
\br
definition
\br
alternative\opt
\br
\terminal{(}
sample-rule-name
\terminal{)}
\end{bnf}

\ssclause{Tables}
{\bfseries
This is an advanced topic,
not for the unwary.
}

\begin{table}[hb]
\centering
\begin{tabular}{|l|l|}
\hline
A&B long%
\\ \hline
C longer&D%
\\ \hline
\end{tabular}
\caption{
Table caption
}
\end{table}

\sclause{Text-level normative elements}
\ssclause{\ttfamily\textbackslash tcode}
The
\tcode{for}
keyword.

\ssclause{\ttfamily\textbackslash defn}
This paragraph defines this
\defn{sample term};
it is automatically added to the index.

\ssclause{\texttt{\$} (math)}
A reference to the meta-variable $X$.

\ssclause{\ttfamily\textbackslash index}
There is a sample index entry
\index{sample index entry}
pointing to this paragraph.

\sclause{Non-normative text}
\ssclause{\ttfamily\textbackslash footnote}
Footnotes should be used very sparingly.
(They're a technical challenge for TeX,
so there are restrictions on their content.)%
\footnote{
This is a sample footnote.
}

\begin{comment}
Supposedly this is a comment.
\end{comment}

\ssclause{\ttfamily\textbackslash begin\{note\}}

A numbered note should generally be used for non-normative text.

\begin{note}
This is a numbered note.
\end{note}

\ssclause{\ttfamily\textbackslash begin\{example\}}
\begin{example}
This is a numbered example.
\end{example}

\ssclause{\ttfamily\textbackslash begin\{verbatim\}}
\begin{verbatim}
int main()
{
    printf("Hello, world\n");
}
\end{verbatim}

\ssclause{\ttfamily\textbackslash begin\{anote\}}

When there is only one note (or example) in a subclause,
it can be changed to an unnumbered note (or example).

\begin{anote}
This is an unnumbered note.
\end{anote}

\ssclause{\ttfamily\textbackslash begin\{anexample\}}
\begin{anexample}
This is an unnumbered example.
\end{anexample}

\sclause{Revisions}
\ssclause{\ttfamily\textbackslash added}
There is
\added{%
some text}
in this paragraph that has been added.

\ssclause{\ttfamily\textbackslash removed}
There is
\removed{%
some text}
in this paragraph that has been removed.

\ssclause{\ttfamily\textbackslash ednote}
\ednote{This a note from the editor.}

\sclause{Constructs not recommended}
\ssclause{\ttfamily\textbackslash begin\{itemize\}}
Lists with unnumbered bullets, especially with more than three items,
should be used sparingly.

\begin{itemize}
\item
This is an itemized list.
\begin{itemize}
\item
This is a nested itemized list.
\end{itemize}
\end{itemize}

\ssclause{\ttfamily\textbackslash begin\{quote\}}
Quote blocks should not be used solely for indentation.
\begin{quote}
This is a quote block.
\end{quote}

\ssclause{\ttfamily\textbackslash begin\{quotation\}}
Quotation blocks should not be used solely for indentation.
\begin{quotation}
This is a quotation block.
\end{quotation}

\ssclause{\ttfamily\textbackslash texttt, \textbackslash ttfamily}
Fixed-width text should be marked with
\verb`\`\verb`begin{verbatim}`
or
\verb`\tcode`
instead.

{\ttfamily
This text is set in typewriter font.
}

\ssclause{\texttt{\textbackslash emph, \textbackslash em,
\textbackslash textit, \textbackslash itshape} (italics)}
Italics for the definition of a term are provided by
\verb`\defn`.
(Italics should never be used when referring to a defined term.)

Italics are the default presentation in a grammar rule,
and should not be made explicit.

A reference to a meta-variable in a normative rule
should be marked as math
(i.e. surrounded by dollar signs).

It should not be necessary to use italics for emphasis.

For a reference to a grammar non-terminal,
italics are used in the C++ standard,
but not the C standard.
When necessary, \verb`\grammarterm` should be used.


\bibannex
\begin{references}
\reference{}{Intel\copyright{} Cilk\texttrademark{} Plus
Language Extension Specification,}
{Intel Corporation}:
\isourl{https://www.cilkplus.org/sites/default/files/open_specifications/Intel_Cilk_plus_lang_spec_1.2.htm}
\reference{}{OpenMP Application Program Interface,}
{OpenMP Architecture Review Board}:
\isourl{http://www.openmp.org/mp-documents/OpenMP4.0.0.pdf}
\end{references}

\printindex
\end{document}