% Use of LuaTeX seems to be recommended.
% This stuff is all about "standard" LaTex capabilities used in the document.
\documentclass[
	techspec,	% for a Technical Specification
	final,		% needed for PDF bookmarks
	notcopyright,	% no copyright statements (yet)
	letterpaper	% for now
	]{isov2}
\let\ifpdf\relaxed	% let hyperref implement \ifpdf
\usepackage[
	pdftitle={Extensions for parallel programming},
	pdfauthor={Clark Nelson},
	pdfsubject={WORKING DRAFT Technical Specification},
	pdfkeywords={},
	pdfstartview=FitH,
	colorlinks=true, % not boxed
	linkcolor=blue,	% usual default for browsers
	urlcolor=blue
	]{hyperref}	% for PDF hyperlinks and bookmarks
\usepackage{chngcntr}	% for paragraph numbering
\usepackage{color}	% for editorial marking
\usepackage[T1]{fontenc} % to use 256-character fonts
\usepackage{fontspec}	% to disable ligatures
\usepackage{ifthen} 	% for conditional code/text
\usepackage{listings}	% for code listings
\usepackage{makeidx}	% for indexing
\usepackage[normalem]{ulem}	% for underlining and strikeout
\usepackage{underscore} % to simplify use of underscores
\usepackage{verbatim}	% for code font
\usepackage{xspace}	% to simplify macros that expand to text
\makeindex
\setcounter{secnumdepth}{6}
\setcounter{tocdepth}{2} % practical TOC size
\setcounter{tocdepth}{6} % uncomment to check outline structure

% These macros are from the WG21 standard source.
\input{macros}
\renewcommand{\indextext}[1]{\index{#1}}
\renewcommand{\indexgram}[1]{\index{#1}}

% This is for macros invented for/by CPLEX.
\newboolean{hidecpp}
\setboolean{hidecpp}{false}
\newcommand{\docdate}[3]{
\newcommand{\Year}{#1}
\newcommand{\Mo}{#2}
\newcommand{\Dy}{#3}
}
% cpp environment for C++-specific text
\ifthenelse{\boolean{hidecpp}}
{\let\cpp=\comment
\let\endcpp=\endcomment
\newcommand{\nocpp}[1]{#1}
\newcommand{\yescpp}[1]{}
}
{\newenvironment{cpp}
{\sffamily\slshape[C++:}
{]}
\newcommand{\nocpp}[1]{}
\newcommand{\yescpp}[1]{#1}
}
% Prefixed names
\newcommand{\prefix}{cplex}
\newcommand{\pfx}[1]{\tcode{\prefix{}\_#1}}
\newcommand{\pfxdefn}[1]{\pfx{#1}\indextext{\idxcode{#1}}}
%% Paragraph numbering
\newcounter{Paras}
\counterwithin{Paras}{clause}
\counterwithin{Paras}{sclause}
\counterwithin{Paras}{ssclause}
\counterwithin{Paras}{sssclause}
\counterwithin{Paras}{ssssclause}
\counterwithin{Paras}{sssssclause}
\makeatletter
\def\pnum{\addtocounter{Paras}{1}\noindent\llap{{%
  \scriptsize\raisebox{.7ex}{\arabic{Paras}}}\hspace{\@totalleftmargin}\quad}}
\renewcommand{\definition}[2]{\@defcl{#1}\index{#1} #2}
\makeatother


% These parameters, and the documentclass options,
% are about document identification and overall formatting.
\docdate{2016}{03}{10}
\newcommand{\wgdocno}{N2017}
\newcommand{\epptsno}{\placeholder{EPPTS}}
\standard{\epptsno}
\yearofedition{2018}
\languageofedition{(E)}
\renewcommand{\extrahead}{\Year-\Mo-\Dy{} \wgdocno}

\begin{document}

\begin{cover}
\begin{tabular}{l l}
\textbf{Revision:} & \extrahead \\
\textbf{Reply to:} & Clark Nelson
\end{tabular}

{\huge
\bfseries
Programming language C --- \\
Extensions for parallel programming --- \\
Part 1: Thread-based parallelism
}

{\large
The contents of this document match
the Working Draft of the CPLEX study group adopted on 2016-03-07.
Only the document identification and title page have been changed.

The consensus of the study group is that this document is ready
to be transferred to WG14 for further processing
to produce a Technical Specification.
However, the study group does not believe its work is done.

This document describes only parallel execution
that uses multi-processor/multi-core technology.
The study group believes that further work needs to be done
to support SIMD/vector technology.
A goal is that a SIMD parallel loop could be expressed
in terms almost identical to a parallel loop in this document,
just by varying a keyword or two.
Arithmetic on array sections, much as in Fortran,
could also provide better support for SIMD technology.
Support for GPGPU technology is also desirable,
but may be more of a challenge.

The title of this document has been changed
to better align the expectations of the reader
with those of the study group.
However, it should be understood that full planning
for an ISO/IEC multipart Technical Specification ---
including determining the title and scope of each part in advance ---
has not yet been done.
}
\clearpage
\end{cover}

% The table of contents appears here.

\begin{foreword}
\input{tspasfwdbp}

ISO/TS
\epptsno{}
was prepared by Technical Committee ISO/IEC JTC1/SC22/WG14.
\footnote{FYI:
This is the only paragraph in the Foreword that has anything in it
that's not just boilerplate.
}

\fwdnopatents
\end{foreword}

\begin{introduction}
\intropatents
\end{introduction}

\title{Programming languages}{C}{Extensions for parallel programming}

\scopeclause
\begin{inscope}{technical specification}
\item
Extensions to the C language to simplify writing a parallel program.
\end{inscope}
\begin{outofscope}{technical specification}
\item
Support for writing a concurrent program.
\end{outofscope}

\normrefsclause
\normrefbp{technical specification}
\begin{nreferences}
\isref{ISO/IEC 9899:2011(E)}{Programming languages --- C}
\isref{ISO/IEC 14882:2014(E)}{Programming languages --- C++}
\end{nreferences}

\defclause
\begin{definitions}

For the purposes of this document,
the following terms and definitions apply.

\definition{thread of execution}{
flow of control within a program,
including a top-level statement or expression,
and recursively including every function invocation it executes%
\footnote{FYI:
Adapted from the C++ standard.
}
}

\definition{OS thread}{
service provided by an operating system,
which can be invoked using the
\tcode{thrd_create} function,
for executing multiple threads of execution concurrently
}

\begin{note}
There is typically significant overhead involved
in creating a new OS thread.
\end{note}

\definition{thread}{
thread of execution, or OS thread
}

\begin{note}
This word, when used without qualification, is ambiguous.
\end{note}

\definition{execution agent}{
entity, such as an OS thread,
that may execute a thread of execution
in parallel with other execution agents%
\footnote{FYI:
Adapted from the C++ standard.
}
}

\definition{task}{
thread of execution within a program
that can be correctly executed asynchronously
with respect to
independent threads of execution from}
the program

\definition{concurrent program}{
program that uses multiple concurrent interacting threads of execution,
each with its own progress requirements
}

\begin{example}
A program that has separate server and client threads
is a concurrent program.
\end{example}

\definition{parallel program}{
program whose computation
involves independent tasks,
which may be distributed across multiple computational units
to be executed simultaneously
}

\begin{note}
If sufficient computational resources are available,
a parallel program may execute significantly faster than
an otherwise equivalent serial program.
\end{note}

\end{definitions}


\clause{Document conventions}

\pnum
This source and issue list for this document are hosted at
\isourl{https://github.com/wg14-cplex/epp}.

\pnum
\begin{cpp}
Text that is specific to C++
is enclosed in square brackets
and presented in oblique sans-serif type.
\end{cpp}

\pnum
Definitions of terms and grammar non-terminals defined in the C
\begin{cpp}
or C++
\end{cpp}
standard are not duplicated in this document.
Terms and grammar non-terminals defined in this document
are referenced in the index.
The ``cplex_'' prefix of library identifiers
is omitted from the index entry.

\pnum
According to the ISO editing directives,
the use of footnotes
``shall be kept to a minimum.''
Almost all of the footnotes in this document
are not intended to survive to final publication.
Most footnotes are classified by an abbreviation:
\begin{description}
\item[FYI]
A point of information.
\item[DFEP]
Departure from existing practice.
\end{description}

\begin{comment}
\pnum
Annex A contains information
concerning the editing of the LaTeX source of this document.
It will not survive to final publication.
\end{comment}

\clause{Predefined macro names}

\pnum
The following macro name is defined by the implementation:

\begin{description}
\item[\tcode{__STDC_PARALLEL_EXT__}]
The integer constant \tcode{\Year\Mo}.
\end{description}

\clause{Task execution}

\pnum
A task is permitted to execute
either in the invoking thread
or in a worker thread implicitly created by the implementation.
Independent tasks executing in the same thread
are indeterminately sequenced with respect to one another.
Independent tasks executing in different threads
are unsequenced with respect to one another.

\pnum
When execution of an independent task completes,
execution
\defn{joins}
with its parent task.
The completion of a task synchronizes
with the completion of the associated task block,
or with the next execution of a sync statement
within the associated task block.

\pnum
It is unspecified whether a worker thread is reused
for multiple tasks
during the execution of a program.
The lifetimes (creation and termination points)
of worker threads are unspecified.
An attempt by the program
to terminate, detach or join with 
a worker thread
results in undefined behavior.

\pnum
\added{%
The mapping of tasks to worker threads is unspecified,
except that the code immediately following a task block
is guaranteed to be executed by the same thread
as that which executed the code
immediately preceding that task block.
The mapping of a task to a worker thread is stable,
except at spawn or join points.%
}%
\footnote{%
Should the statements of a task block be allowed to execute
on a different thread than the one that entered the task block?
}

\begin{note}
\added{%
This means, for example,
that a mutex acquired outside of a task block
can be neither acquired nor released within the task block,
since there is no assurance that such an action
would occur on the same thread (for unlock)
or on a different thread (for lock).
}
\end{note}	


\clause{Reduction types}
\sclause{Introduction}

\pnum
A
\defn{reduction type}
describes a member object with a particular type,
called the
\defn{proxied type},
and an associated combiner operation,
along with other optional aspects,
to support common parallel computations.

\pnum
Attempting to access an object with reduction type
and either thread or allocated storage duration
results in undefined behavior.

\sclause{Reduction specifiers}
\ssclause*{Syntax}

\begin{bnf}
\nontermdef{reduction-specifier}
\br
\terminal{_Reduction} identifier\opt{} \terminal{\{} reduction-aspect-list \terminal{\}}
\br
\terminal{_Reduction} identifier
\end{bnf}

\begin{bnf}
\nontermdef{reduction-aspect-list}
\br
reduction-aspect
\br
reduction-aspect-list \terminal{,} reduction-aspect
\end{bnf}

\begin{bnf}
\nontermdef{reduction-aspect}
\br
\terminal{_Type} \terminal{:} type-name
\br
\terminal{_Combiner} \terminal{:} combiner-operation
\br
\terminal{_Initializer} \terminal{:} initializer
\br
\terminal{_Finalizer} \terminal{:} constant-expression
\br
\terminal{_Order} \terminal{:} reduction-order-constraint
\end{bnf}

\begin{bnf}
\nontermdef{combiner-operation}
\br
constant-expression
\br
builtin-combiner-operation
\end{bnf}

\begin{bnf}
\nontermdef{builtin-combiner-operation}
\textnormal{one of}
\br
\terminal{*=} \terminal{+=}
\br
\terminal{\&=} \terminal{\textasciicircum=} \terminal{|=}
\br
\terminal{_And} \terminal{_Or}
\br
\terminal{_Min} \terminal{_Max}
\br
\terminal{_Last}
\end{bnf}

\begin{bnf}
\nontermdef{reduction-order-constraint}
\br
\terminal{_Commutative}
\br
\terminal{_Associative}
\end{bnf}

\ssclause*{Constraints}

\pnum
The type and combiner aspects shall be present in every reduction specifier.
Any given kind of aspect shall not be present more than once
in a reduction specifier. 

\pnum
The proxied type of a reduction shall be
one of the following:
an unqualified arithmetic type,
an unqualified pointer to object type,
or an unqualified complete structure or union type.

\pnum
If the combiner is a constant expression,
then it shall be an address constant
referring to a function taking two arguments,
both of pointer-to-proxied type.
If it is a compound assignment operator,
then it shall be one
for which the constraints of the corresponding combination method
are satisfied using lvalues having the proxied type.%
\footnote{
For example, if the proxied type is a floating type,
the operator shall not be
\tcode{|=}.
}
If it is
\tcode{_And}
or
\tcode{_Or},
the proxied type shall be an integer type.
If it is
\tcode{_Min}
or
\tcode{_Max},
the proxied type shall be a real arithmetic type.

\pnum
The initializer of a reduction
shall be suitable to initialize
an object of the proxied type
having static storage duration,
or shall be an address constant referring to a function.
If it is an address constant,
it shall refer to a function
taking one argument of pointer-to-proxied type.

\pnum
The finalizer of a reduction
shall be an address constant referring to a function
taking one argument of pointer-to-proxied type.

\ssclause*{Semantics}

\pnum
A reduction is a type containing a proxied member
of an associated proxied type.
Each concurrently-executing task
that refers to an object of reduction type
has its own distinct proxied member object
(called its
\defn{view})
of the object.

\begin{note}
Thus, when they are used as intended,
reduction objects can be updated from different tasks
without causing data races.
\end{note}

\pnum
At any point within a parallel computation,
the value of a view reflects a sub-computation on the reduction object.
At some point after the completion of a set of tasks,
partial results are combined, two at a time,
using the combiner operation of the reduction type
to merge one view into another.
The resulting value reflects the union
of the sub-computations on the two original views.
 
\pnum
A reduction object is
\defn{serially consistent}
when no other task that could access it in parallel is executing.
A serially consistent reduction object has a single view,
called the
\defn{root view},
reflecting the entire set of computations on the reduction object
since its creation.
 
\pnum
If the combiner operation is a function pointer,
the combination is performed by executing:

\begin{bnf}
\terminal{(*}combiner\terminal{)(\&}into_view\terminal{, \&}from_view\terminal{);}
\end{bnf}

Otherwise the combination is performed according to
\tref{tab:comb}.
In all cases the object designated
$into\_view$
is expected to be modified to reflect the combined sub-computations.
The object designated
$from\_view$
is unused after being combined with
$into\_view$
except as the argument to the finalizer.
\footnote{TODO:
How does
\tcode{FENV_ACCESS}
affect reductions?
}

\begin{table}[ht]
\caption{%
Combination method for built-in combiners
}
\label{tab:comb}
\centering
\begin{tabular}{|l|l|}
\hline
\bfseries Specified combiner&
\bfseries Combination method
\\ \hline
\tcode{*=}&
$into\_view$ \tcode{*=} $from\_view$ \tcode{;}
\\ \hline
\tcode{+=}&
$into\_view$ \tcode{+=} $from\_view$ \tcode{;}
\\ \hline
\tcode{\&=}&
$into\_view$ \tcode{\&=} $from\_view$ \tcode{;}
\\ \hline
\tcode{\textasciicircum=}&
$into\_view$ \tcode{\textasciicircum=} $from\_view$ \tcode{;}
\\ \hline
\tcode{|=}&
$into\_view$ \tcode{|=} $from\_view$ \tcode{;}
\\ \hline
\tcode{_And}&
$into\_view$ \tcode{=} $into\_view$ \tcode{\&\&} $from\_view$ \tcode{;}
\\ \hline
\tcode{_Or}&
$into\_view$ \tcode{=} $into\_view$ \tcode{||} $from\_view$ \tcode{;}
\\ \hline
\tcode{_Min}&
\tcode{if (} $from\_view$ \tcode{<} $into\_view$ \tcode{)}
$into\_view$ \tcode{=} $from\_view$ \tcode{;}
\\ \hline
\tcode{_Max}&
\tcode{if (} $from\_view$ \tcode{>} $into\_view$ \tcode{)}
$into\_view$ \tcode{=} $from\_view$ \tcode{;}
\\ \hline
\tcode{_Last}&
$into\_view$ \tcode{=} $from\_view$ \tcode{;}
\\ \hline
\end{tabular}
\end{table}

\pnum
At some unspecified point
before a task refers to a reduction object for the first time,
the view used by the task is allocated, and initialized
using the initializer of the reduction's type.
If the initializer is a function pointer,
the initialization is performed by executing:

\begin{bnf}
\terminal{(*}initializer\terminal{)(\&}view\terminal{);}
\end{bnf}

Otherwise, the view is initialized
as if it were an object with static storage duration,
using the specified initializer.
If the initializer is not specified,
and the specified combiner is in
\tref{tab:init},
the view is initialized with the corresponding value from
\tref{tab:init}.
\footnote{TODO:
It's not clear that what we have is complete
considering dynamically-allocated reduction objects.
}

\begin{table}[ht]
\caption{%
Default initializers for built-in combiners
}
\label{tab:init}
\centering
\begin{tabular}{|l|l|}
\hline
\bfseries Specified combiner&
\bfseries Default initializer
\\ \hline
\tcode{*=}&
the value 1, converted to the proxied type
\\ \hline
\tcode{\&=}&
the bitwise complement of converting zero to the proxied type
\\ \hline
\tcode{_And}&
the value 1, converted to the proxied type
\\ \hline
\tcode{_Min}&
the maximum value representable by the proxied type
\\ \hline
\tcode{_Max}&
the minimum value representable by the proxied type
\\ \hline
\end{tabular}
\end{table}

\pnum
It is unspecified whether a view is created
for a task that does not access a specific reduction object.
A task's view of a reduction object can be shared with other tasks
that do not execute concurrently;
it is not necessary that each task have a distinct view.
Any initializer function is invoked only once for each distinct view,
regardless how many tasks share that view.
The initializer aspect of a reduction type is not used
to initialize the root view of a reduction object;
for purposes of initialization,
the root view behaves like a member of the reduction object. 

\pnum
If the type of a reduction object has a finalizer,
after a view has been used as the
$from\_view$
argument to the combiner operation,
the view is finalized by executing:

\begin{bnf}
\terminal{(*}finalizer\terminal{)(\&}view\terminal{);}
\end{bnf}

The finalizer is not applied to the root view.

\begin{note}
A single view is never passed to concurrent invocations
of the initializer, combiner operation, or finalizer
of a reduction object.
\end{note}

\pnum
Views are presented to the combiner operation in pairings
that depend on the order aspect of the reduction type.
In the following,
for any point
$P$
at which the reduction object is serially consistent,
let
$S$
represent the sequence of modifications
that would be applied to the reduction object
in the serialization of the program:

\begin{description}
\item[\tcode{_Commutative}]

View combinations can be paired arbitrarily.
The value of the root view at
$P$
reflects all of the operations in
$S$,
but applied in an unspecified order.

\begin{note}
Operations that are sensitive to operand order
(e.g., string append)
or to operation grouping
(e.g., addition in the presence of overflow)
might yield nondeterministic results that differ from the serialization.
\end{note}

\item[\tcode{_Associative}]

Views are presented to the combiner operation
such that the values of
$into\_view$
and
$from\_view$
reflect consecutive subsequences of
$S$,
respectively called
$SL$ and $SR$.
The value computed by the combiner operation
(stored in
$into\_view$)
reflects the concatenation of
$SL$ and $SR$,
which comprises a contiguous subsequence of
$S$.
The value of the root view at
$P$
reflects all of the operations in
$S$,
but applied in unspecified groupings.

\begin{note}
Operations that are sensitive to operation grouping
(e.g., addition in the presence of overflow)
might yield nondeterministic results that differ from the serialization. 
\end{note}

\begin{comment}
\item[\tcode{\removed{_Serial}}]
\footnote{TODO:
Neither Cilk nor OpenMP provides this sort of guarantee.
Should CPLEX include it?
}
Views are presented to the combiner operation
such that the value of
$into\_view$
is the partial result reflecting an initial subsequence of
$S$,
called
$SL$,
and the value of
$from\_view$
is the partial result reflecting the next
(single) consecutive operation in
$S$
following
$SL$.
Thus, modifications are made to a single view
(ultimately the root view)
in the same order as for the serialization.

\begin{note}
It is likely that the implementation
will need to suppress concurrent execution of tasks
in order to ensure this (fully deterministic) modification order.
If so, there need not be more than one view,
and the combiner operation might not be invoked at all. 
\end{note}
\end{comment}
\end{description}

\pnum
If the combiner aspect of a reduction type is
\tcode{_Last},
the default for the order aspect is
\tcode{_Associative}.
Otherwise, the default for the order aspect is
\tcode{_Commutative}.%
\footnote{DFEP:
Neither OpenMP nor Cilk supports specifying the order constraint for reduction.
OpenMP reductions provide only the guarantees of
\tcode{_Commutative};
Cilk reductions provide the guarantees of
\tcode{_Associative}.
}

\pnum
Two reduction types declared in separate translation units
are compatible if all of the following conditions are satisfied:

\begin{enumerate}
\item
Neither is declared with a tag, or they are declared with the same tag.
\item
Their proxied types are compatible.
\item
Their combiner operations are the same
(either the same builtin combiner operation or the same function).
\item
Neither specifies a finalizer,
or their finalizers are specified with equal values.
\item
Their order aspects are specified or defaulted to be the same.
\item
If either is specified with an initializer that is the address of a function,
then the other is specified with an initializer
that is the address of the same function;
otherwise, corresponding scalar components of the proxied type
are initialized with equal values,
and in corresponding components with union type,
members with compatible types are initialized.
\end{enumerate}

\sclause{Reduction conversions}

\pnum
An lvalue with reduction type is implicitly converted,
through a run-time view lookup,
to an lvalue with its corresponding proxied type.
This conversion is suppressed
if the address of the lvalue is taken in a context
where the result is immediately converted,
implicitly or explicitly,
to a pointer to the original reduction type.

\begin{example}
Consider this code:

\begin{verbatim}
_Reduction int_add { _Type: int, _Combiner: += };
_Reduction int_add x, y;
x = y;	// int assignment: both operands converted by view lookup
void f(_Reduction int_add *, int *);
f(&x, &y);  // view lookup performed on y, but not on x
int *pi = &x;
f(pi, &x);  // error
\end{verbatim}

The last line of the example is an error because
\tcode{pi},
as an expression,
is not taking the address of a reduction-converted lvalue.
The expression that takes that address is in the previous line.
The reduction lvalue conversion can be suppressed during translation,
but not necessarily reversed during execution.
As a further example:

\begin{verbatim}
f((_Reduction int_add *)pi, &x);
\end{verbatim}

This is not an error,
but the first argument passed to the function
need not point to the reduction object,
so undefined behavior results if it used as if it did.
\end{example}

\sclause{Integration with the C standard}

Change paragraph 7 of subclause 6.2.1 ``Scopes of identifiers":

\begin{quote}
Structure, union,
\added{reduction,}
and enumeration tags have scope that begins
just after the appearance of the tag
in a type specifier that declares the tag. ...
\end{quote}

Change the list item of paragraph 1
of subclause 6.2.3 ``Name spaces of identifiers":

\begin{quote}
\begin{itemize}
\item
the
\textit{tags}
of structures, unions,
\added{reductions,}
and enumerations
(disambiguated by following any of the keywords
\tcode{struct},
\tcode{union},
\tcode{\added{_Reduction}},
or
\tcode{enum});
\end{itemize}
\end{quote}

Add a new item
to the list in paragraph 20 of subclause 6.2.5 ``Types":

\begin{quote}
\begin{itemize}
\item
A
\defn{reduction type}
describes a member object with a particular type,
called the
\defn{proxied type},
and an associated combiner operation,
along with other optional aspects,
to support common parallel computations.
\end{itemize}
\end{quote}

Change the grammar rule in subclause 6.4.1 ``Keywords",
by adding new alternatives:

\begin{quote}
\begin{bnf}
\nontermdef{keyword}
\textnormal{one of}\br
...\br
\terminal{\added{_Reduction}}\br
\terminal{\added{_Task}}\br
\terminal{\added{_Block}}\br
\terminal{\added{_Spawn}}\br
\terminal{\added{_Sync}}\br
\terminal{\added{_Call}}\br
\terminal{\added{_Capture}}\br
\terminal{\added{_Options}}
\end{bnf}
\end{quote}

Add a new item to the list
in paragraph 1 of subclause 6.5.16.1 ``Simple assignment":

\begin{quote}
\begin{itemize}
\item
the left operand has atomic, qualified, or unqualified
pointer to some reduction type,
and the right operand expression
is the taking of the address
of some object having a qualified or unqualified version
of the same reduction type
(whose type is therefore a pointer to the reduction's proxied type),
and the type pointed to by the left
has all the qualifiers of the type pointed to by the right;
\end{itemize}
\end{quote}

Change the grammar rule in subclause 6.7.2 ``Type specifiers",
by adding a new alternative:

\begin{quote}
\begin{bnf}
\nontermdef{type-specifier}
\br
...
\br
\added{reduction-specifier}
\end{bnf}
\end{quote}

Change paragraphs 2 through 6 of subclause 6.7.2.3 ``Tags":

\begin{quote}
Where two declarations that use the same tag declare the same type,
they shall both use the same choice of
\tcode{struct},
\tcode{union},
\tcode{\added{_Reduction}},
or
\tcode{enum}.

A type specifier of the form

\begin{bnf}
\terminal{\added{_Reduction}} \added{identifier}
\end{bnf}

\added{without a reduction aspect list, or}

\begin{bnf}
\terminal{enum} identifier
\end{bnf}

without an enumerator list
shall only appear after the type it specifies is complete.

All declarations of structure, union,
\added{reduction,}
or enumerated types that have the same scope and use the same tag
declare the same type.

Two declarations of structure, union,
\added{reduction,}
or enumerated types which are in different scopes or use different tags
declare distinct types.
Each declaration of a structure, union,
\added{reduction,}
or enumerated type which does not include a tag declares a distinct type.

A type specifier of the form

\begin{bnf}
struct-or-union identifier\opt{} \terminal{\{} struct-declaration-list \terminal{\}}
\end{bnf}

\added{or}

\begin{bnf}
\terminal{\added{_Reduction}} \added{identifier\opt{} \terminal{\{} reduction-aspect-list \terminal{\}}}
\end{bnf}

or

\begin{bnf}
\terminal{enum} identifier\opt{} \terminal{\{} enumerator-list \terminal{\}}
\end{bnf}

or

\begin{bnf}
\terminal{enum} identifier\opt{} \terminal{\{} enumerator-list \terminal{, \}}
\end{bnf}

declares a structure, union,
\added{reduction,}
or enumerated type.
The list defines the structure content, union content,
\added{reduction content,}
or enumeration content. ...
\end{quote}

Change paragraph 9 of 6.7.2.3 ``Tags":

\begin{quote}
If a type specifier of the form

\begin{bnf}
struct-or-union identifier
\end{bnf}

\added{or}

\begin{bnf}
\terminal{\added{_Reduction}} \added{identifier}
\end{bnf}

or

\begin{bnf}
\terminal{enum} identifier
\end{bnf}

occurs other than as part of one of the above forms,
and a declaration of the identifier as a tag is visible,
then it specifies the same type as that other declaration,
and does not redeclare the tag.
\end{quote}

Change paragraph 1 of subclause 6.7.6.3
``Function declarations (including prototypes)":

\begin{quote}
A function declarator shall not specify a return type
that is a function type or an array type
\added{or a reduction type}.
\end{quote}

Add a new paragraph following paragraph 8 of subclause 6.7.6.3
``Function declarations (including prototypes)":

\begin{quote}
A declaration of a parameter as a reduction type
shall be adjusted to be a pointer to the same reduction type.
\end{quote}

Add a new paragraph following paragraph 8
of subclause 6.7.9 ``Initializers":

\begin{quote}
Any initializer for an object of reduction type initializes its root view.
\footnote{TODO:
At the time it was incorporated into this document,
the reduction proposal added:
"If an object of reduction type is not initialized explicitly,
then the root view is initialized as every other view is initialized."
But this conflicted with the direction CPLEX approved (I think).
}
\end{quote}

\begin{cpp}
\sclause{Integration with the C++ standard}

Add new entries to table 3 in subclause 2.11 ``Keywords'':
\footnote{TODO:
This section is currently just an outline, and needs to be filled in.
}

Change paragraph 3 of subclause 3.3.2 ``Point of declaration":

Changes to subclause 3.4.4 ``Elaborated type specifiers":

Add a new item to the list in paragraph 1
of subclause 3.9.2 ``Compound types":

Add a new paragraph following paragraph 3
of subclause 4.10 ``Pointer conversions":

Change the grammar rule in paragraph 1 of subclause 7.1.6 ``Type specifiers":

Change paragraphs 2 and 3 of subclause 7.1.6.3 ``Elaborated type specifiers":

Change paragraph 5 of subclause 8.3.5 ``Functions":

Change to subclause 8.5 ``Initializers":

\end{cpp}


\clause{Captures}

\sclause{Introduction}

\pnum
A spawn capture allows a spawn statement
to make a copy of a variable
prior to the start of asynchronous execution.
A reduction capture allows a task block or parallel loop
to temporarily associate a reduction object
with an existing object,
to simplify parallel computation of a reduction.

\sclause{Spawn captures}

\ssclause*{Syntax}

\begin{bnf}
\nontermdef{spawn-capture}
\br
\terminal{_Capture} \terminal{(} spawn-capture-list \terminal{)}
\end{bnf}

\begin{bnf}
\nontermdef{spawn-capture-list}
\br
spawn-capture-item
\br
spawn-capture-list \terminal{,} spawn-capture-item
\end{bnf}

\begin{bnf}
\nontermdef{spawn-capture-item}
\br
identifier
\br
identifier \terminal{=} expression
\end{bnf}

\ssclause*{Constraints}

\pnum
If no expression is present in a spawn capture item,
the identifier shall be a name that is already in scope
at the beginning of the spawn capture item,
and the effective expression is taken to be the same as the identifier.
Otherwise, the effective expression
is the expression in the spawn capture item.

\pnum
The effective expression shall have complete object type.

\ssclause*{Semantics}

\pnum
Each spawn capture item declares a new object
named by the item's identifier,
having automatic storage duration.
The type of the declared object is that of the effective expression.
The scope of the name extends from the end of the spawn capture item
until the end of the spawn statement with which it is associated.

\pnum
The declared object is initialized
with the value of the effective expression.
The initialization of the declared object
occurs before asynchronous execution
of the spawned compound statement.

\pnum
Change the first sentence of paragraph 3 of subclause 6.3.2.1:

\begin{quote}
Except when it
\added{is the effective expression in a spawn capture item, or}
is the operand of the \tcode{sizeof} operator,
the \tcode{_Alignof} operator,
or the unary \tcode{\&} operator,
or is a string literal used to initialize an array,
an expression that has type ``array of type''
is converted to an expression with type ``pointer to type''
that points to the initial element of the array object
and is not an lvalue. ...
\end{quote}

\sclause{Reduction captures}

\ssclause*{Syntax}

\begin{bnf}
\nontermdef{reduction-capture}
\br
\terminal{_Reduction} \terminal{(} reduction-capture-list \terminal{)}
\end{bnf}

\begin{bnf}
\nontermdef{reduction-capture-list}
\br
reduction-capture-item
\br
reduction-capture-list \terminal{,} reduction-capture-item
\end{bnf}

\begin{bnf}
\nontermdef{reduction-capture-item}
\br
declaration-specifiers declarator
\br
declaration-specifiers declarator \terminal{:} expression
\end{bnf}

\ssclause*{Constraints}

\pnum
The declaration specifiers in a reduction capture item
shall specify a reduction type,
and shall not specify static or thread storage duration.

\pnum
If no expression is present in a reduction capture item,
the identifier in the declarator
shall be a name that is already in scope
at the beginning of the reduction capture item,
and the effective expression is taken to be the same as the identifier.
Otherwise, the effective expression
is the expression in the reduction capture item.

\pnum
The effective expression shall be a modifiable lvalue,
and shall have a type that is compatible with the proxied type
of the item's reduction type.

\ssclause*{Semantics}

\pnum
Each reduction capture item declares a new object,
named by the identifier in the declarator.
The type of the object is the type specified by the declaration specifiers.
The scope of the name extends from the end of the reduction capture item
until the end of the task block or loop with which it is associated.

\pnum
Before execution of the task block or loop,
the new reduction object is initialized
with the value of the object designated by the effective expression.
Upon completion of the task block or loop,
the value of the reduction object is assigned back
to the object designated by the effective expression.

\pnum
Change the first sentence of paragraph 2 of subclause 6.3.2.1:

\begin{quote}
Except when it
\added{is the expression in a reduction capture item, or}
is the operand of the \tcode{sizeof} operator,
the \tcode{_Alignof} operator, the
unary \tcode{\&} operator,
the \tcode{++} operator,
the \tcode{--} operator,
or the left operand of the \tcode{.} operator
or an assignment operator,
an lvalue that does not have array type
is converted to the value stored in the designated object
(and is no longer an lvalue);
this is called
\defn{lvalue conversion}. ...
\end{quote}


\include{countable}

\clause{Parallel loops}

\pnum
A
\defn{parallel loop}
is a
\tcode{for}
statement with loop qualifiers.
The grammar of the iteration statement (6.8.5, paragraph 1)
is modified to read:

\begin{bnf}
\nontermdef{iteration-statement}
\br
\terminal{while} \terminal{(} expression \terminal{)} statement
\br
\terminal{do} statement \terminal{while} \terminal{(} expression \terminal{)} \terminal{;}
\br
loop-qualifiers\opt{} \terminal{for} \terminal{(}
expression\opt{} \terminal{;}
expression\opt{} \terminal{;}
expression\opt{} \terminal{)} statement
\br
loop-qualifiers\opt{} \terminal{for} \terminal{(}
declaration
expression\opt{} \terminal{;}
expression\opt{} \terminal{)} statement
\end{bnf}

\begin{cpp}
The grammar of
\nonterminal{iteration-statement}
(6.5 [stmt.iter], paragraph 1)
is modified to read:

\begin{bnf}
\nontermdef{iteration-statement}
\br
\terminal{while} \terminal{(} expression \terminal{)} statement
\br
\terminal{do} statement \terminal{while} \terminal{(} expression \terminal{)} \terminal{;}
\br
loop-qualifiers\opt{} \terminal{for} \terminal{(}
for-init-statement
condition\opt{} \terminal{;}
expression\opt{} \terminal{)} statement
\br
loop-qualifiers\opt{} \terminal{for} \terminal{(}
for-range-declaration \terminal{:}
for-range-initializer \terminal{)} statement
\end{bnf}

\end{cpp}

\pnum
The following rules are added to the grammar:

\begin{bnf}
\nontermdef{loop-qualifiers}
\br
\terminal{_Task} qualifier-clauses\opt
\end{bnf}

\begin{bnf}
\nontermdef{qualifier-clauses}
\br
loop-parameters qualifier-clauses\opt
\br
reduction-capture qualifier-clauses\opt
\end{bnf}

\begin{bnf}
\nontermdef{loop-parameters}
\br
\terminal{_Options} \terminal{(} expression \terminal{)}
\end{bnf}

\pnum
A parallel loop is a counted loop,
and shall satisfy all the constraints of a counted loop.

\pnum
In a parallel loop with the
\tcode{_Task}
loop qualifier,
each iteration is
executed as a task, independent of
all other iterations of that execution of the loop.
At the end of the loop,
execution joins with all of these tasks.
\footnote{TODO:
Should there be some constraint(s) on references
from the body of a parallel loop
to the name of an object declared outside the loop?
}

\pnum
If loop parameters are specified as part of the loop qualifiers,
the contained expression shall have type
``pointer to \pfx{loop_params_t}'',
as defined in header
\tcode{<cplex.h>}.%
\footnote{DFEP:
This syntax for specifying tuning parameters for a loop
is a CPLEX invention.
}

\pnum
The
\defn{serialization}
of a parallel loop
is obtained by deleting the loop qualifiers from the loop.

\clause{Task statements}
\sclause{Introduction}

\pnum
The grammar of a statement (6.8, paragraph 1)
\begin{cpp}
(clause 6, paragraph 1)
\end{cpp}
is modified to add task-statement as a new alternative.

\ssclause*{Syntax}

\begin{bnf}
\nontermdef{task-statement}
\br
task-block-statement
\br
task-spawn-statement
\br
task-sync-statement
\br
task-call-statement
\end{bnf}

\sclause{The task block statement}
\ssclause*{Syntax}

\begin{bnf}
\nontermdef{task-block-statement}
\br
\terminal{_Task} \terminal{_Block} reduction-capture\opt{} compound-statement
\end{bnf}

\ssclause*{Constraints}

\pnum
There shall be no
\tcode{switch}
or jump statement that might transfer control into or out of
a task block statement.

\ssclause*{Semantics}

\pnum
Defines a task block, within which tasks can be spawned.
At the end of the contained compound statement,
execution joins with
all child tasks spawned directly or indirectly
within the compound statement.

\pnum
For a given statement, the
\defn{associated task block}
is defined as follows.
For a statement within a task spawn statement,
there is no associated task block,
except within a nested task block statement
or parallel loop.
For a statement within a task block statement
or parallel loop,
the associated task block is the smallest enclosing task block statement
or parallel loop.
Otherwise, for a statement within the body of a function
declared with the spawning function specifier,
the associated task block is the same as it was
at the point of the task spawning call statement
that invoked the spawning function.
For a statement in any other context,
there is no associated task block.

\begin{note}
Task blocks can be nested lexically and/or dynamically.
Determination of the associated task block is a hybrid process:
lexically within a function,
and dynamically across calls to spawning functions.%
\footnote{DFEP:
In Cilk, this determination can be done entirely lexically.
In OpenMP, this determination can be done entirely dynamically.
}
Code designated for execution in another thread
by means other than a task statement
(e.g. using
\tcode{thrd_create})
is not part of any task block.
\end{note}

\pnum
Attempting to terminate a task block with
\tcode{longjmp}
produces undefined behavior.

\sclause{The task spawn statement}
\ssclause*{Syntax}

\begin{bnf}
\nontermdef{task-spawn-statement}
\br
\terminal{_Task} \terminal{_Spawn} spawn-capture\opt{} compound-statement
\end{bnf}

\ssclause*{Constraints}

\pnum
A task spawn statement shall have an associated task block.

\pnum
There shall be no
\tcode{switch}
or jump statement that might transfer control into or out of
a task spawn statement.

\pnum
Attempting to terminate a task spawn statement with
\tcode{longjmp}
produces undefined behavior.

\ssclause*{Semantics}
\pnum
The contained compound statement is executed as a task,
independent of the continued execution
of the associated task block.

\sclause{The task sync statement}
\ssclause*{Syntax}

\begin{bnf}
\nontermdef{task-sync-statement}
\br
\terminal{_Task} \terminal{_Sync} \terminal{;}
\end{bnf}

\ssclause*{Constraints}

\pnum
A task sync statement shall have an associated task block.

\ssclause*{Semantics}
\pnum
Execution joins with
all child tasks of the associated task block
of the task sync statement.

\sclause{The task spawning call statement}
\ssclause*{Syntax}
\begin{bnf}
\nontermdef{task-call-statement}
\br
\terminal{_Task} \terminal{_Call} expression-statement
\end{bnf}
\ssclause*{Constraints}

\pnum
A task spawning call statement shall have an associated task block.

\ssclause*{Semantics}

\pnum
The contained expression statement is executed normally.
Any called spawning function is allowed to spawn tasks;
any such tasks are associated with the associated task block
of the task spawning call statement,
and are
independent of
the statements of the task block
following the task spawning call statement.

\begin{note}
A call to a task spawning function need not be
the ``outermost'' operation of the expression statement.
A task spawning call statement
might invoke more than one spawning function,
or might invoke none.
\end{note}

\sclause{The spawning function specifier}
\ssclause*{Syntax}
\pnum
A new alternative is added to the grammar of function specifier
(6.7.4 paragraph 1):

\begin{bnf}
\nontermdef{function-specifier}
\br
\terminal{_Task} \terminal{_Call}
\end{bnf}

\ssclause*{Constraints}
\pnum
If a spawning function specifier appears
on any declaration of a function,
it shall appear on every declaration of that function.
A function declared with a spawning function specifier
shall be called only from a task spawning call statement.


\clause{Parallel loop hint parameters \tcode{\textless{}cplex.h\textgreater{}}}

\sclause{Introduction}

\pnum
The header
\tcode{\textless cplex.h\textgreater}
defines several types and several macros.

\pnum
The
\pfxdefn{loop_params_t}
type is a structure type
with an unspecified number of members
for specifying parameters for tuning hints for a parallel loop.
A program whose output
depends on the value specified for any tuning hint parameter
is not considered a correct program.

\begin{note}
There is no guarantee that setting any tuning hint parameter
will improve the performance of the program.
\end{note}

\pnum
The
\pfxdefn{sched_kind_t}
type is an enumerated type
with at least the following enumeration constants,
each with nonzero value:

\begin{ttfamily}
\pfxdefn{sched_static}\\
\pfxdefn{sched_dynamic}\\
\pfxdefn{sched_guided}
\end{ttfamily}

\pnum
The
\pfxdefn{workload_t}
type is an enumerated type
with at least the following enumeration constants,
each with nonzero value:

\begin{ttfamily}
\pfxdefn{workload_balanced}\\
\pfxdefn{workload_unbalanced}
\end{ttfamily}

\pnum
The
\pfxdefn{affinity_t}
type is an enumerated type
with at least the following enumeration constants,
each with nonzero value:

\begin{ttfamily}
\pfxdefn{affinity_close}\\
\pfxdefn{affinity_spread}
\end{ttfamily}

\pnum
When an object of type
\pfx{loop_params_t}
is used as the loop parameter of a parallel loop,
the loop is described as being associated with the object.
If the associated object is modified during the execution of the loop,
the behavior is undefined.
When executing a parallel loop associated with an object of type
\pfx{loop_params_t},
for any parameter for which the corresponding member has the value zero,
an unspecified default value is used.

\pnum
Each parameter is represented by a pair of macros:
one to set the value of the parameter in the parameter block,
and one to get the value of the parameter from the parameter block.

\begin{note}
Because these methods are specified as macros,
not functions,
taking the address of any of them need not be supported.
However, an implementation is also free
to provide functions with these names.
\end{note}

\begin{example}
Hint parameters for a parallel loop can be specified as follows:
\begin{verbatim}
#include <cplex.h>
#include <stdlib.h>
int main(int argc, char *argv[])
{
    cplex_loop_params_t hints = { 0 };
    if (argc > 1) {
        cplex_set_num_threads(&hints, atoi(argv[1]));
    }
    cplex_set_chunk_size(&hints, 1000);
    _Task _Options(&hints) for (long i = 0; i < 1000000; i++) {
        do_something_with(i);
    }
}
\end{verbatim}
\end{example}

\sclause{The \tcode{num_threads} parameter}
\ssclause*{Synopsis}
\begin{ttfamily}
\#include <cplex.h>\\
void \pfxdefn{set_num_threads}(\pfx{loop_params_t} *hints, int num_threads);\\
int \pfx{get_num_threads}(\pfx{loop_params_t} *hints);
\end{ttfamily}

\ssclause*{Description}
\pnum
The
\pfx{set_num_threads}
macro sets to
\tcode{num_threads}
the recommended number of execution agents to be used
to execute the iterations of a parallel loop
associated with the object pointed to by
\tcode{hints}.

\sclause{The \tcode{chunk_size} parameter}
\ssclause*{Synopsis}
\begin{ttfamily}
\#include <cplex.h>\\
void \pfxdefn{set_chunk_size}(\pfx{loop_params_t} *hints, int chunk_size);\\
int \pfx{get_chunk_size}(\pfx{loop_params_t} *hints);
\end{ttfamily}

\ssclause*{Description}
\pnum
The
\pfx{set_chunk_size}
macro sets to
\tcode{chunk_size}
the recommended maximum number of iterations
of a parallel loop associated with the object pointed to by
\tcode{hints}
to be grouped together to be executed sequentially
in a single thread of execution.

\sclause{The \tcode{schedule_kind} parameter}
\ssclause*{Synopsis}
\begin{ttfamily}
\#include <cplex.h>\\
void \pfxdefn{set_schedule_kind}(\pfx{loop_params_t} *hints,\\
\hspace*{2.5in}\pfx{sched_kind_t} kind);\\
\pfx{sched_kind_t} \pfx{get_schedule_kind}(\pfx{loop_params_t} *hints);
\end{ttfamily}

\ssclause*{Description}
\pnum
The
\pfx{set_schedule_kind}
macro sets to
\tcode{kind}
the recommended scheduling algorithm
for a parallel loop associated with the object pointed to by
\tcode{hints}.

\begin{note}
Setting the
\tcode{schedule_kind}
parameter to a particular value
may (but need not)
select the corresponding OpenMP loop-scheduling algorithm.
\end{note}

\sclause{The \tcode{workload_balance} parameter}
\ssclause*{Synopsis}
\begin{ttfamily}
\#include <cplex.h>\\
void \pfxdefn{set_workload_balance}(\pfx{loop_params_t} *hints,\\
\hspace*{2.5in}\pfx{workload_t} kind);\\
\pfx{workload_t} \pfx{get_workload_balance}(\pfx{loop_params_t} *hints);
\end{ttfamily}

\ssclause*{Description}
\pnum
The
\pfx{set_workload_balance}
macro sets to
\tcode{kind}
the workload-balancing characteristic
for a parallel loop associated with the object pointed to by
\tcode{hints}.

\pnum
For a loop with a balanced workload,
each iteration should be assumed to execute
in approximately the same amount of time.
A loop with an unbalanced workload
should be assumed to have iterations
taking widely varying amounts of time.

\begin{note}
This parameter is semantically a statement about the associated loop,
whereas the
\tcode{schedule_kind}
parameter is semantically a request to the implementation.
Setting this parameter to
\pfx{workload_balanced}
may have an effect similar to setting the schedule to
\pfx{schedule_static}.
Setting this parameter to
\pfx{workload_unbalanced}
may have an effect similar to setting the schedule to
\pfx{schedule_dynamic}
or
\pfx{schedule_guided}.
\end{note}

\sclause{The \tcode{affinity} parameter}
\ssclause*{Synopsis}
\begin{ttfamily}
\#include <cplex.h>\\
void \pfxdefn{set_affinity}(\pfx{loop_params_t} *hints, \pfx{affinity_t} kind);\\
\pfx{affinity_t} \pfx{get_affinity}(\pfx{loop_params_t} *hints);
\end{ttfamily}

\ssclause*{Description}
\pnum
The
\pfx{set_affinity}
macro sets to
\tcode{kind}
the recommended affinity
for a parallel loop associated with the object pointed to by
\tcode{hints}.

\pnum
The affinity of a loop indicates whether the loop benefits
from being executed by co-located hardware threads,
or whether performance is likely to improve if the software threads
are spread over multiple cores.


%\%include{samples}

\bibannex
\begin{references}
\reference{}{Intel\copyright{} Cilk\texttrademark{} Plus
Language Extension Specification,}
{Intel Corporation}:
\isourl{https://www.cilkplus.org/sites/default/files/open_specifications/Intel_Cilk_plus_lang_spec_1.2.htm}
\reference{}{OpenMP Application Program Interface,}
{OpenMP Architecture Review Board}:
\isourl{http://www.openmp.org/mp-documents/OpenMP4.0.0.pdf}
\end{references}

% we definitely want a two-column index
% but I can't figure out how to get one
\printindex
\end{document}
