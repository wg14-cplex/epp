\clause{Reduction types}
\sclause{Introduction}

\pnum
A
\defn{reduction type}
describes a member object with a particular type,
called the
\defn{proxied type},
and an associated combiner operation,
along with other optional aspects,
to support common parallel computations.

\pnum
Attempting to access an object with reduction type
and either thread or allocated storage duration
results in undefined behavior.

\sclause{Reduction specifiers}
\ssclause*{Syntax}

\begin{bnf}
\nontermdef{reduction-specifier}
\br
\terminal{_Reduction} identifier\opt{} \terminal{\{} reduction-aspect-list \terminal{\}}
\br
\terminal{_Reduction} identifier
\end{bnf}

\begin{bnf}
\nontermdef{reduction-aspect-list}
\br
reduction-aspect
\br
reduction-aspect-list \terminal{,} reduction-aspect
\end{bnf}

\begin{bnf}
\nontermdef{reduction-aspect}
\br
\terminal{_Type} \terminal{:} type-name
\br
\terminal{_Combiner} \terminal{:} combiner-operation
\br
\terminal{_Initializer} \terminal{:} initializer
\br
\terminal{_Finalizer} \terminal{:} constant-expression
\br
\terminal{_Order} \terminal{:} reduction-order-constraint
\end{bnf}

\begin{bnf}
\nontermdef{combiner-operation}
\br
constant-expression
\br
builtin-combiner-operation
\end{bnf}

\begin{bnf}
\nontermdef{builtin-combiner-operation}
\textnormal{one of}
\br
\terminal{*=} \terminal{+=}
\br
\terminal{\&=} \terminal{\textasciicircum=} \terminal{|=}
\br
\terminal{_And} \terminal{_Or}
\br
\terminal{_Min} \terminal{_Max}
\br
\terminal{_Last}
\end{bnf}

\begin{bnf}
\nontermdef{reduction-order-constraint}
\br
\terminal{_Commutative}
\br
\terminal{_Associative}
\end{bnf}

\ssclause*{Constraints}

\pnum
The type and combiner aspects shall be present in every reduction specifier.
Any given kind of aspect shall not be present more than once
in a reduction specifier. 

\pnum
The proxied type of a reduction shall be
one of the following:
an unqualified arithmetic type,
an unqualified pointer to object type,
or an unqualified complete structure or union type.

\pnum
If the combiner is a constant expression,
then it shall be an address constant
referring to a function taking two arguments,
both of pointer-to-proxied type.
If it is a compound assignment operator,
then it shall be one
for which the constraints of the corresponding combination method
are satisfied using lvalues having the proxied type.%
\footnote{
For example, if the proxied type is a floating type,
the operator shall not be
\tcode{|=}.
}
If it is
\tcode{_And}
or
\tcode{_Or},
the proxied type shall be an integer type.
If it is
\tcode{_Min}
or
\tcode{_Max},
the proxied type shall be a real arithmetic type.

\pnum
The initializer of a reduction
shall be suitable to initialize
an object of the proxied type
having static storage duration,
or shall be an address constant referring to a function.
If it is an address constant,
it shall refer to a function
taking one argument of pointer-to-proxied type.

\pnum
The finalizer of a reduction
shall be an address constant referring to a function
taking one argument of pointer-to-proxied type.

\ssclause*{Semantics}

\pnum
A reduction type is a type containing a proxied member
of an associated proxied type.
Each concurrently-executing task
that refers to an object of reduction type
has its own distinct proxied member object
(called its
\defn{view})
of the object.

\begin{note}
Thus, when they are used as intended,
reduction objects can be updated from different tasks
without causing data races.
\end{note}

\pnum
At any point within a parallel computation,
the value of a view reflects a sub-computation on the reduction object.
At some point after the completion of a set of tasks,
partial results are combined, two at a time,
using the combiner operation of the reduction type
to merge one view into another.
The resulting value reflects the union
of the sub-computations on the two original views.
 
\pnum
A reduction object is
\defn{serially consistent}
when no other task that could access it in parallel is executing.
A serially consistent reduction object has a single view,
called the
\defn{root view},
reflecting the entire set of computations on the reduction object
since its creation.
 
\pnum
If the combiner operation is a function pointer,
the combination is performed by executing:

\begin{bnf}
\terminal{(*}combiner\terminal{)(\&}into_view\terminal{, \&}from_view\terminal{);}
\end{bnf}

Otherwise the combination is performed according to
\tref{tab:comb}.
In all cases the object designated
$into\_view$
is expected to be modified to reflect the combined sub-computations.
The object designated
$from\_view$
is unused after being combined with
$into\_view$
except as the argument to the finalizer.

\begin{note}
There is no guarantee which thread will execute the combiner
between two concurrent tasks.
As a result, use of thread-local state by a combiner
is not fully reliable.
The floating-point environment
(\tcode{<fenv.h>})
is an example of thread-local state.
Therefore, to reliably use the floating-point environment
with combiner operations,
a copy of the floating-point environment should be incorporated
into the proxied type of the reduction.
At the beginning of the combiner function,
the floating-point environment should be saved in a local object,
and loaded from the destination view.
At the end of the combiner function,
the floating-point environment should be saved
back to the destination view,
and reloaded from the local object.
\end{note}

\begin{table}[ht]
\caption{%
Combination method for built-in combiners
}
\label{tab:comb}
\centering
\begin{tabular}{|l|l|}
\hline
\bfseries Specified combiner&
\bfseries Combination method
\\ \hline
\tcode{*=}&
$into\_view$ \tcode{*=} $from\_view$ \tcode{;}
\\ \hline
\tcode{+=}&
$into\_view$ \tcode{+=} $from\_view$ \tcode{;}
\\ \hline
\tcode{\&=}&
$into\_view$ \tcode{\&=} $from\_view$ \tcode{;}
\\ \hline
\tcode{\textasciicircum=}&
$into\_view$ \tcode{\textasciicircum=} $from\_view$ \tcode{;}
\\ \hline
\tcode{|=}&
$into\_view$ \tcode{|=} $from\_view$ \tcode{;}
\\ \hline
\tcode{_And}&
$into\_view$ \tcode{=} $into\_view$ \tcode{\&\&} $from\_view$ \tcode{;}
\\ \hline
\tcode{_Or}&
$into\_view$ \tcode{=} $into\_view$ \tcode{||} $from\_view$ \tcode{;}
\\ \hline
\tcode{_Min}&
\tcode{if (} $from\_view$ \tcode{<} $into\_view$ \tcode{)}
$into\_view$ \tcode{=} $from\_view$ \tcode{;}
\\ \hline
\tcode{_Max}&
\tcode{if (} $from\_view$ \tcode{>} $into\_view$ \tcode{)}
$into\_view$ \tcode{=} $from\_view$ \tcode{;}
\\ \hline
\tcode{_Last}&
$into\_view$ \tcode{=} $from\_view$ \tcode{;}
\\ \hline
\end{tabular}
\end{table}

\pnum
At some unspecified point
before a task refers to a reduction object for the first time,
the view used by the task is allocated and initialized.
For purposes of initialization,
the root view behaves like a member of the reduction object.
Every other view is initialized
using the initializer of the reduction's type.
If the initializer is a function pointer,
the initialization is performed by executing:

\begin{bnf}
\terminal{(*}initializer\terminal{)(\&}view\terminal{);}
\end{bnf}

Otherwise, the view is initialized
as if it were an object with static storage duration,
using the specified initializer.
If the initializer is not specified,
and the specified combiner is in
\tref{tab:init},
the view is initialized with the corresponding value from
\tref{tab:init}.

\begin{table}[ht]
\caption{%
Default initializers for built-in combiners
}
\label{tab:init}
\centering
\begin{tabular}{|l|l|}
\hline
\bfseries Specified combiner&
\bfseries Default initializer
\\ \hline
\tcode{*=}&
the value 1, converted to the proxied type
\\ \hline
\tcode{\&=}&
the bitwise complement of converting zero to the proxied type
\\ \hline
\tcode{_And}&
the value 1, converted to the proxied type
\\ \hline
\tcode{_Min}&
the maximum value representable by the proxied type
\\ \hline
\tcode{_Max}&
the minimum value representable by the proxied type
\\ \hline
\end{tabular}
\end{table}

\pnum
It is unspecified whether a view is created
for a task that does not access a specific reduction object.
A task's view of a reduction object can be shared with other tasks
that do not execute concurrently;
it is not necessary that each task have a distinct view.
Any initializer function is invoked only once for each distinct view,
regardless how many tasks share that view.

\pnum
If the type of a reduction object has a finalizer,
after a view has been used as the
$from\_view$
argument to the combiner operation,
the view is finalized by executing:

\begin{bnf}
\terminal{(*}finalizer\terminal{)(\&}view\terminal{);}
\end{bnf}

The finalizer is not applied to the root view.

\begin{note}
A single view is never passed to concurrent invocations
of the initializer, combiner operation, or finalizer
of a reduction object.
\end{note}

\pnum
Views are presented to the combiner operation in pairings
that depend on the order aspect of the reduction type.
In the following,
for any point
$P$
at which the reduction object is serially consistent,
let
$S$
represent the sequence of modifications
that would be applied to the reduction object
in the serialization of the program:

\begin{description}
\item[\tcode{_Commutative}]

View combinations can be paired arbitrarily.
The value of the root view at
$P$
reflects all of the operations in
$S$,
but applied in an unspecified order.

\begin{note}
Operations that are sensitive to operand order
(e.g., string append)
or to operation grouping
(e.g., addition in the presence of overflow)
might yield nondeterministic results that differ from the serialization.
\end{note}

\item[\tcode{_Associative}]

Views are presented to the combiner operation
such that the values of
$into\_view$
and
$from\_view$
reflect consecutive subsequences of
$S$,
respectively called
$SL$ and $SR$.
The value computed by the combiner operation
(stored in
$into\_view$)
reflects the concatenation of
$SL$ and $SR$,
which comprises a contiguous subsequence of
$S$.
The value of the root view at
$P$
reflects all of the operations in
$S$,
but applied in unspecified groupings.

\begin{note}
Operations that are sensitive to operation grouping
(e.g., addition in the presence of overflow)
might yield nondeterministic results that differ from the serialization. 
\end{note}

\begin{comment}
\item[\tcode{\removed{_Serial}}]
\footnote{TODO:
Neither Cilk nor OpenMP provides this sort of guarantee.
Should CPLEX include it?
}
Views are presented to the combiner operation
such that the value of
$into\_view$
is the partial result reflecting an initial subsequence of
$S$,
called
$SL$,
and the value of
$from\_view$
is the partial result reflecting the next
(single) consecutive operation in
$S$
following
$SL$.
Thus, modifications are made to a single view
(ultimately the root view)
in the same order as for the serialization.

\begin{note}
It is likely that the implementation
will need to suppress concurrent execution of tasks
in order to ensure this (fully deterministic) modification order.
If so, there need not be more than one view,
and the combiner operation might not be invoked at all. 
\end{note}
\end{comment}
\end{description}

\pnum
If the combiner aspect of a reduction type is
\tcode{_Last},
the default for the order aspect is
\tcode{_Associative}.
Otherwise, the default for the order aspect is
\tcode{_Commutative}.%
\footnote{DFEP:
Neither OpenMP nor Cilk supports specifying the order constraint for reduction.
OpenMP reductions provide only the guarantees of
\tcode{_Commutative};
Cilk reductions provide the guarantees of
\tcode{_Associative}.
}

\pnum
Two reduction types declared in separate translation units
are compatible if all of the following conditions are satisfied:

\begin{enumerate}
\item
Neither is declared with a tag, or they are declared with the same tag.
\item
Their proxied types are compatible.
\item
Their combiner operations are the same
(either the same builtin combiner operation or the same function).
\item
Neither specifies a finalizer,
or their finalizers are specified with equal values.
\item
Their order aspects are specified or defaulted to be the same.
\item
If either is specified with an initializer that is the address of a function,
then the other is specified with an initializer
that is the address of the same function;
otherwise, corresponding scalar components of the proxied type
are initialized with equal values,
and in corresponding components with union type,
members with compatible types are initialized.
\end{enumerate}

\sclause{Reduction conversions}

\pnum
An lvalue with reduction type is implicitly converted,
through a run-time view lookup,
to an lvalue with its corresponding proxied type.
This conversion is suppressed
if the address of the lvalue is taken in a context
where the result is immediately converted,
implicitly or explicitly,
to a pointer to the original reduction type.

\begin{example}
Consider this code:

\begin{verbatim}
_Reduction int_add { _Type: int, _Combiner: += };
_Reduction int_add x, y;
x = y;	// int assignment: both operands converted by view lookup
void f(_Reduction int_add *, int *);
f(&x, &y);  // view lookup performed on y, but not on x
int *pi = &x;
f(pi, &x);  // error
\end{verbatim}

The last line of the example is an error because
\tcode{pi},
as an expression,
is not taking the address of a reduction-converted lvalue.
The expression that takes that address is in the previous line.
The reduction lvalue conversion can be suppressed during translation,
but not necessarily reversed during execution.
As a further example:

\begin{verbatim}
f((_Reduction int_add *)pi, &x);
\end{verbatim}

This is not an error,
but the first argument passed to the function
need not point to the reduction object,
so undefined behavior results if it used as if it did.
\end{example}

\sclause{Integration with the C standard}

Change paragraph 7 of subclause 6.2.1 ``Scopes of identifiers":

\begin{quote}
Structure, union,
\added{reduction,}
and enumeration tags have scope that begins
just after the appearance of the tag
in a type specifier that declares the tag. ...
\end{quote}

Change the list item of paragraph 1
of subclause 6.2.3 ``Name spaces of identifiers":

\begin{quote}
\begin{itemize}
\item
the
\textit{tags}
of structures, unions,
\added{reductions,}
and enumerations
(disambiguated by following any of the keywords
\tcode{struct},
\tcode{union},
\tcode{\added{_Reduction}},
or
\tcode{enum});
\end{itemize}
\end{quote}

Add a new item
to the list in paragraph 20 of subclause 6.2.5 ``Types":

\begin{quote}
\begin{itemize}
\item
A
\defn{reduction type}
describes a member object with a particular type,
called the
\defn{proxied type},
and an associated combiner operation,
along with other optional aspects,
to support common parallel computations.
\end{itemize}
\end{quote}

Change the grammar rule in subclause 6.4.1 ``Keywords",
by adding new alternatives:

\begin{quote}
\begin{bnf}
\nontermdef{keyword}
\textnormal{one of}\br
...\br
\terminal{\added{_Reduction}}\br
\terminal{\added{_Task}}\br
\terminal{\added{_Block}}\br
\terminal{\added{_Spawn}}\br
\terminal{\added{_Sync}}\br
\terminal{\added{_Call}}\br
\terminal{\added{_Capture}}\br
\terminal{\added{_Options}}
\end{bnf}
\end{quote}

Add a new item to the list
in paragraph 1 of subclause 6.5.16.1 ``Simple assignment":

\begin{quote}
\begin{itemize}
\item
the left operand has atomic, qualified, or unqualified
pointer to some reduction type,
and the right operand expression
is the taking of the address
of some object having a qualified or unqualified version
of the same reduction type
(whose type is therefore a pointer to the reduction's proxied type),
and the type pointed to by the left
has all the qualifiers of the type pointed to by the right;
\end{itemize}
\end{quote}

Change the grammar rule in subclause 6.7.2 ``Type specifiers",
by adding a new alternative:

\begin{quote}
\begin{bnf}
\nontermdef{type-specifier}
\br
...
\br
\added{reduction-specifier}
\end{bnf}
\end{quote}

Change paragraphs 2 through 6 of subclause 6.7.2.3 ``Tags":

\begin{quote}
Where two declarations that use the same tag declare the same type,
they shall both use the same choice of
\tcode{struct},
\tcode{union},
\tcode{\added{_Reduction}},
or
\tcode{enum}.

A type specifier of the form

\begin{bnf}
\terminal{\added{_Reduction}} \added{identifier}
\end{bnf}

\added{without a reduction aspect list, or}

\begin{bnf}
\terminal{enum} identifier
\end{bnf}

without an enumerator list
shall only appear after the type it specifies is complete.

All declarations of structure, union,
\added{reduction,}
or enumerated types that have the same scope and use the same tag
declare the same type.

Two declarations of structure, union,
\added{reduction,}
or enumerated types which are in different scopes or use different tags
declare distinct types.
Each declaration of a structure, union,
\added{reduction,}
or enumerated type which does not include a tag declares a distinct type.

A type specifier of the form

\begin{bnf}
struct-or-union identifier\opt{} \terminal{\{} struct-declaration-list \terminal{\}}
\end{bnf}

\added{or}

\begin{bnf}
\terminal{\added{_Reduction}} \added{identifier\opt{} \terminal{\{} reduction-aspect-list \terminal{\}}}
\end{bnf}

or

\begin{bnf}
\terminal{enum} identifier\opt{} \terminal{\{} enumerator-list \terminal{\}}
\end{bnf}

or

\begin{bnf}
\terminal{enum} identifier\opt{} \terminal{\{} enumerator-list \terminal{, \}}
\end{bnf}

declares a structure, union,
\added{reduction,}
or enumerated type.
The list defines the structure content, union content,
\added{reduction content,}
or enumeration content. ...
\end{quote}

Change paragraph 9 of 6.7.2.3 ``Tags":

\begin{quote}
If a type specifier of the form

\begin{bnf}
struct-or-union identifier
\end{bnf}

\added{or}

\begin{bnf}
\terminal{\added{_Reduction}} \added{identifier}
\end{bnf}

or

\begin{bnf}
\terminal{enum} identifier
\end{bnf}

occurs other than as part of one of the above forms,
and a declaration of the identifier as a tag is visible,
then it specifies the same type as that other declaration,
and does not redeclare the tag.
\end{quote}

Change paragraph 1 of subclause 6.7.6.3
``Function declarations (including prototypes)":

\begin{quote}
A function declarator shall not specify a return type
that is a function type or an array type
\added{or a reduction type}.
\end{quote}

Add a new paragraph following paragraph 8 of subclause 6.7.6.3
``Function declarations (including prototypes)":

\begin{quote}
A declaration of a parameter as a reduction type
shall be adjusted to be a pointer to the same reduction type.
\end{quote}

Add a new paragraph following paragraph 8
of subclause 6.7.9 ``Initializers":

\begin{quote}
Any initializer for an object of reduction type initializes its root view.
\end{quote}

\begin{cpp}
\sclause{Integration with the C++ standard}

Add new entries to table 3 in subclause 2.11 ``Keywords'':

Change paragraph 3 of subclause 3.3.2 ``Point of declaration":

Changes to subclause 3.4.4 ``Elaborated type specifiers":

Add a new item to the list in paragraph 1
of subclause 3.9.2 ``Compound types":

Add a new paragraph following paragraph 3
of subclause 4.10 ``Pointer conversions":

Change the grammar rule in paragraph 1 of subclause 7.1.6 ``Type specifiers":

Change paragraphs 2 and 3 of subclause 7.1.6.3 ``Elaborated type specifiers":

Change paragraph 5 of subclause 8.3.5 ``Functions":

Change to subclause 8.5 ``Initializers":

\end{cpp}
