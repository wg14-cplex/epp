\defclause
\begin{definitions}

For the purposes of this document,
the following terms and definitions apply.

\definition{thread of execution}{
flow of control within a program,
including a top-level statement or expression,
and recursively including every function invocation it executes%
\footnote{FYI:
Adapted from the C++ standard.
}
}

\definition{OS thread}{
service provided by an operating system,
which can be invoked using the
\tcode{thrd_create} function,
for executing multiple threads of execution concurrently
}

\begin{note}
There is typically significant overhead involved
in creating a new OS thread.
\end{note}

\definition{thread}{
thread of execution, or OS thread
}

\begin{note}
This word, when used without qualification, is ambiguous.
\end{note}

\definition{execution agent}{
entity, such as an OS thread,
that may execute a thread of execution
in parallel with other execution agents%
\footnote{FYI:
Adapted from the C++ standard.
}
}

\definition{task}{
thread of execution within a program
that can be correctly executed asynchronously
with respect to
independent threads of execution from}
the program

\definition{concurrent program}{
program that uses multiple concurrent interacting threads of execution,
each with its own progress requirements
}

\begin{example}
A program that has separate server and client threads
is a concurrent program.
\end{example}

\definition{parallel program}{
program whose computation
involves independent tasks,
which may be distributed across multiple computational units
to be executed simultaneously
}

\begin{note}
If sufficient computational resources are available,
a parallel program may execute significantly faster than
an otherwise equivalent serial program.
\end{note}

\end{definitions}
