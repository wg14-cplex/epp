\clause{Parallel loops}

\pnum
A
\defn{parallel loop}
is a
\tcode{for}
statement with loop qualifiers.
The grammar of the iteration statement (6.8.5, paragraph 1)
is modified to read:

\begin{bnf}
\nontermdef{iteration-statement}
\br
\terminal{while} \terminal{(} expression \terminal{)} statement
\br
\terminal{do} statement \terminal{while} \terminal{(} expression \terminal{)} \terminal{;}
\br
loop-qualifiers\opt{} \terminal{for} \terminal{(}
expression\opt{} \terminal{;}
expression\opt{} \terminal{;}
expression\opt{} \terminal{)} statement
\br
loop-qualifiers\opt{} \terminal{for} \terminal{(}
declaration
expression\opt{} \terminal{;}
expression\opt{} \terminal{)} statement
\end{bnf}

\begin{cpp}
The grammar of
\nonterminal{iteration-statement}
(6.5 [stmt.iter], paragraph 1)
is modified to read:

\begin{bnf}
\nontermdef{iteration-statement}
\br
\terminal{while} \terminal{(} expression \terminal{)} statement
\br
\terminal{do} statement \terminal{while} \terminal{(} expression \terminal{)} \terminal{;}
\br
loop-qualifiers\opt{} \terminal{for} \terminal{(}
for-init-statement
condition\opt{} \terminal{;}
expression\opt{} \terminal{)} statement
\br
loop-qualifiers\opt{} \terminal{for} \terminal{(}
for-range-declaration \terminal{:}
for-range-initializer \terminal{)} statement
\end{bnf}

\end{cpp}

\pnum
The following rules are added to the grammar:

\begin{bnf}
\nontermdef{loop-qualifiers}
\br
\terminal{_Task} qualifier-clauses\opt
\end{bnf}

\begin{bnf}
\nontermdef{qualifier-clauses}
\br
loop-parameters qualifier-clauses\opt
\br
reduction-capture qualifier-clauses\opt
\end{bnf}

\begin{bnf}
\nontermdef{loop-parameters}
\br
\terminal{_Options} \terminal{(} expression \terminal{)}
\end{bnf}

\pnum
A parallel loop is a counted loop,
and shall satisfy all the constraints of a counted loop.

\pnum
In a parallel loop with the
\tcode{_Task}
loop qualifier,
each iteration is
executed as a task, independent of
all other iterations of that execution of the loop.
At the end of the loop,
execution joins with all of these tasks.
\footnote{TODO:
Should there be some constraint(s) on references
from the body of a parallel loop
to the name of an object declared outside the loop?
}

\pnum
If loop parameters are specified as part of the loop qualifiers,
the contained expression shall have type
``pointer to \pfx{loop_params_t}'',
as defined in header
\tcode{<cplex.h>}.%
\footnote{DFEP:
This syntax for specifying tuning parameters for a loop
is a CPLEX invention.
}

\pnum
The
\defn{serialization}
of a parallel loop
is obtained by deleting the loop qualifiers from the loop.