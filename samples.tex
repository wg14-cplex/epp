\infannex{LaTeX samples}
\sclause{Block-level normative elements}
\ssclause{\ttfamily\textbackslash begin\{enumerate\}}
\begin{enumerate}
\item
This is an enumerated list.
\begin{enumerate}
\item
This is a nested enumerated list.
\end{enumerate}
\end{enumerate}

\ssclause{\ttfamily\textbackslash begin\{description\}}
\begin{description}
\item[term]
This is a description list.
\item[longer term]
More description list.
\end{description}

\ssclause{\ttfamily\textbackslash begin\{bnf\}}
Grammar rules appear in the index automatically.
The ``opt'' suffix should be applied only to the name of a non-terminal.

\begin{bnf}
\nontermdef{sample-rule-name}
\br
definition
\br
alternative\opt
\br
\terminal{(}
sample-rule-name
\terminal{)}
\end{bnf}

\ssclause{Tables}
{\bfseries
This is an advanced topic,
not for the unwary.
}

\begin{table}[hb]
\centering
\begin{tabular}{|l|l|}
\hline
A&B long%
\\ \hline
C longer&D%
\\ \hline
\end{tabular}
\caption{
Table caption
}
\end{table}

\sclause{Text-level normative elements}
\ssclause{\ttfamily\textbackslash tcode}
The
\tcode{for}
keyword.

\ssclause{\ttfamily\textbackslash defn}
This paragraph defines this
\defn{sample term};
it is automatically added to the index.

\ssclause{\texttt{\$} (math)}
A reference to the meta-variable $X$.

\ssclause{\ttfamily\textbackslash index}
There is a sample index entry
\index{sample index entry}
pointing to this paragraph.

\sclause{Non-normative text}
\ssclause{\ttfamily\textbackslash footnote}
Footnotes should be used very sparingly.
(They're a technical challenge for TeX,
so there are restrictions on their content.)%
\footnote{
This is a sample footnote.
}

\begin{comment}
Supposedly this is a comment.
\end{comment}

\ssclause{\ttfamily\textbackslash begin\{note\}}

A numbered note should generally be used for non-normative text.

\begin{note}
This is a numbered note.
\end{note}

\ssclause{\ttfamily\textbackslash begin\{example\}}
\begin{example}
This is a numbered example.
\end{example}

\ssclause{\ttfamily\textbackslash begin\{verbatim\}}
\begin{verbatim}
int main()
{
    printf("Hello, world\n");
}
\end{verbatim}

\ssclause{\ttfamily\textbackslash begin\{anote\}}

When there is only one note (or example) in a subclause,
it can be changed to an unnumbered note (or example).

\begin{anote}
This is an unnumbered note.
\end{anote}

\ssclause{\ttfamily\textbackslash begin\{anexample\}}
\begin{anexample}
This is an unnumbered example.
\end{anexample}

\sclause{Revisions}
\ssclause{\ttfamily\textbackslash added}
There is
\added{%
some text}
in this paragraph that has been added.

\ssclause{\ttfamily\textbackslash removed}
There is
\removed{%
some text}
in this paragraph that has been removed.

\ssclause{\ttfamily\textbackslash ednote}
\ednote{This a note from the editor.}

\sclause{Constructs not recommended}
\ssclause{\ttfamily\textbackslash begin\{itemize\}}
Lists with unnumbered bullets, especially with more than three items,
should be used sparingly.

\begin{itemize}
\item
This is an itemized list.
\begin{itemize}
\item
This is a nested itemized list.
\end{itemize}
\end{itemize}

\ssclause{\ttfamily\textbackslash begin\{quote\}}
Quote blocks should not be used solely for indentation.
\begin{quote}
This is a quote block.
\end{quote}

\ssclause{\ttfamily\textbackslash begin\{quotation\}}
Quotation blocks should not be used solely for indentation.
\begin{quotation}
This is a quotation block.
\end{quotation}

\ssclause{\ttfamily\textbackslash texttt, \textbackslash ttfamily}
Fixed-width text should be marked with
\verb`\`\verb`begin{verbatim}`
or
\verb`\tcode`
instead.

{\ttfamily
This text is set in typewriter font.
}

\ssclause{\texttt{\textbackslash emph, \textbackslash em,
\textbackslash textit, \textbackslash itshape} (italics)}
Italics for the definition of a term are provided by
\verb`\defn`.
(Italics should never be used when referring to a defined term.)

Italics are the default presentation in a grammar rule,
and should not be made explicit.

A reference to a meta-variable in a normative rule
should be marked as math
(i.e. surrounded by dollar signs).

It should not be necessary to use italics for emphasis.

For a reference to a grammar non-terminal,
italics are used in the C++ standard,
but not the C standard.
When necessary, \verb`\grammarterm` should be used.
